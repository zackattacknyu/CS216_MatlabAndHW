\documentclass[11pt,psfig]{article}
\usepackage{epsfig}
\usepackage{times}
\usepackage{amssymb}
\usepackage{float}

\newcount\refno\refno=1
\def\ref{\the\refno \global\advance\refno by 1}
\def\ux{\underline{x}}
\def\uw{\underline{w}}
\def\bw{\underline{w}}
\def\ut{\underline{\theta}}
\def\umu{\underline{\mu}} 
\def\bmu{\underline{\mu}} 
\def\be{p_e^*}
\newcount\eqnumber\eqnumber=1
\def\eq{\the \eqnumber \global\advance\eqnumber by 1}
\def\eqs{\eq}
\def\eqn{\eqno(\eq)}

 \pagestyle{empty}
\def\baselinestretch{1.1}
\topmargin1in \headsep0.3in
\topmargin0in \oddsidemargin0in \textwidth6.5in \textheight8.5in
\begin{document}
\setlength{\parskip}{1.2ex plus0.3ex minus 0.3ex}


\thispagestyle{empty} \pagestyle{myheadings} \markright{Homework
1: CS 216, Image Understanding: Spring 2014}



\title{CS 216 Homework 1}
\author{Zachary DeStefano, 15247592}
\date{Due Date: April 9}

\maketitle

\vfill\eject

\section*{Problem 4}

\subsection*{Problem 4a}

The variable $x$ will be assigned to an array that is the numbers 1 through 5 arranged in a random order. 

\subsection*{Problem 4b}

The variable $a$ will be assigned to an array that is the numbers 1 through 10 arranged in chronological order. The variable $b$ will end up being assigned an array that is the numbers $1,4,7,10$ because it is the first number and then every third number after that. 

\subsection*{Problem 4c}

The variable $f$ will be assigned to an array that is the numbers 1501 to 2000 in chronological order. The variable $g$ will be assigned to the indices of $f$ where the entries are greater than $1850$ which is $351$ to $500$. The variable $h$ will then be assigned an array that is the numbers $1851$ to $2000$.  

\subsection*{Problem 4d}

The variables $x$ will be assigned to a vector of 10 entries where each entry has value 22. The variable $y$ will be assigned the sum of the entries which will be 220.

\subsection*{Problem 4e}

The variable $a$ will be assigned an array that is the numbers 1 to 1000 in chronological order. The variable $b$ will be assigned a vector that is the numbers 1 to 1000 in reverse order. 

%\begin{figure}[H]
%\centering
%\includegraphics[height=4in]{prob1plot.jpg}
%\caption{Probability of Class Labels with decision boundaries marked}
%\end{figure}


\end{document}








