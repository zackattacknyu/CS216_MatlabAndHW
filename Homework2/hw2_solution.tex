\documentclass[11pt,psfig]{article}
\usepackage{epsfig}
\usepackage{times}
\usepackage{amssymb}
\usepackage{float}

\newcount\refno\refno=1
\def\ref{\the\refno \global\advance\refno by 1}
\def\ux{\underline{x}}
\def\uw{\underline{w}}
\def\bw{\underline{w}}
\def\ut{\underline{\theta}}
\def\umu{\underline{\mu}} 
\def\bmu{\underline{\mu}} 
\def\be{p_e^*}
\newcount\eqnumber\eqnumber=1
\def\eq{\the \eqnumber \global\advance\eqnumber by 1}
\def\eqs{\eq}
\def\eqn{\eqno(\eq)}

 \pagestyle{empty}
\def\baselinestretch{1.1}
\topmargin1in \headsep0.3in
\topmargin0in \oddsidemargin0in \textwidth6.5in \textheight8.5in
\begin{document}
\setlength{\parskip}{1.2ex plus0.3ex minus 0.3ex}


\thispagestyle{empty} \pagestyle{myheadings} \markright{Homework
\#: CS 216, Image Understanding: Spring 2014}



\title{CS 216 Homework 2}
\author{Zachary DeStefano, 15247592}
\date{Due Date: April 25, 2014}

\maketitle

\vfill\eject

\newpage

\section*{Problem 1}

We will prove that $(f*g)*h=f*(g*h)$. First off, let $x=f*g$ and let $y=g*h$. \\
\[
x(t) = (f*g)(t) = \sum_{s=-\infty}^{\infty} f(t-s)g(s)
\]
\[
y(t) = (g*h)(t) = \sum_{s=-\infty}^{\infty} g(s)h(t-s)
\]
This means that
\[
(x*h)(t) = \sum_{v=-\infty}^{\infty} x(v)h(t-v) = \sum_{v=-\infty}^{\infty} \sum_{s=-\infty}^{\infty} f(v-s)g(s)h(t-v)
\]
Similarly
\[
(f*y)(t) = \sum_{v=-\infty}^{\infty} f(t-v)y(v) = \sum_{v=-\infty}^{\infty} \sum_{s=-\infty}^{\infty} f(t-v)g(s)h(v-s)
\]
After a change of variables (TODO: More detail), the two sides are equal. 

\section*{Problem 2}

We will prove that $(f*g)*h \neq f*(g*h)$ if * is correlation. First off, let $x=f*g$ and let $y=g*h$. \\
\[
x(t) = (f*g)(t) = \sum_{s=-\infty}^{\infty} f(t+s)g(s)
\]
\[
y(t) = (g*h)(t) = \sum_{s=-\infty}^{\infty} g(s)h(t+s)
\]
This means that
\[
(x*h)(t) = \sum_{v=-\infty}^{\infty} x(v)h(t+v) = \sum_{v=-\infty}^{\infty} \sum_{s=-\infty}^{\infty} f(v+s)g(s)h(t+v)
\]
Similarly
\[
(f*y)(t) = \sum_{v=-\infty}^{\infty} f(t+v)y(v) = \sum_{v=-\infty}^{\infty} \sum_{s=-\infty}^{\infty} f(t+v)g(s)h(v+s)
\]
After a change of variables (TODO: More detail), the two sides are definitely not equal. 

\section*{Problem 3}

Let $g_1(x) = g_2(y) = N(0,\sigma^2)$. Then
\[
g_1(x) = \frac{1}{\sqrt{2\pi \sigma^2}} e^{-\frac{x^2}{2\sigma^2}}
\]
\[
g_2(y) = \frac{1}{\sqrt{2\pi \sigma^2}} e^{-\frac{y^2}{2\sigma^2}}
\]
This means that
\[
g_1(x)g_2(y) = \frac{1}{2\pi \sigma^2} e^{-\frac{x^2 + y^2}{2\sigma^2}}
\]
This allows us to say that $g(x,y) = g_1(x)g_2(y)$

%\begin{figure}[H]
%\centering
%\includegraphics[height=4in]{prob1plot.jpg}
%\caption{Probability of Class Labels with decision boundaries marked}
%\end{figure}


\end{document}








