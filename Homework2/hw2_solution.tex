\documentclass[11pt,psfig]{article}
\usepackage{epsfig}
\usepackage{times}
\usepackage{amssymb}
\usepackage{float}

\newcount\refno\refno=1
\def\ref{\the\refno \global\advance\refno by 1}
\def\ux{\underline{x}}
\def\uw{\underline{w}}
\def\bw{\underline{w}}
\def\ut{\underline{\theta}}
\def\umu{\underline{\mu}} 
\def\bmu{\underline{\mu}} 
\def\be{p_e^*}
\newcount\eqnumber\eqnumber=1
\def\eq{\the \eqnumber \global\advance\eqnumber by 1}
\def\eqs{\eq}
\def\eqn{\eqno(\eq)}

 \pagestyle{empty}
\def\baselinestretch{1.1}
\topmargin1in \headsep0.3in
\topmargin0in \oddsidemargin0in \textwidth6.5in \textheight8.5in
\begin{document}
\setlength{\parskip}{1.2ex plus0.3ex minus 0.3ex}


\thispagestyle{empty} \pagestyle{myheadings} \markright{Homework
\#: CS 216, Image Understanding: Spring 2014}



\title{CS 216 Homework 2}
\author{Zachary DeStefano, 15247592}
\date{Due Date: April 25, 2014}

\maketitle

\vfill\eject

\newpage

\section*{Problem 1}

We will prove that $(f*g)*h=f*(g*h)$. First off, let $x=f*g$ and let $y=g*h$. \\
\[
x(t) = (f*g)(t) = \sum_{s=-\infty}^{\infty} f(t-s)g(s)
\]
\[
y(t) = (g*h)(t) = \sum_{s=-\infty}^{\infty} g(s)h(t-s)
\]
This means that
\[
(x*h)(t) = \sum_{v=-\infty}^{\infty} x(v)h(t-v) = \sum_{v=-\infty}^{\infty} \sum_{s=-\infty}^{\infty} f(v-s)g(s)h(t-v)
\]
Similarly
\[
(f*y)(t) = \sum_{v=-\infty}^{\infty} f(t-v)y(v) = \sum_{v=-\infty}^{\infty} \sum_{s=-\infty}^{\infty} f(t-v)g(s)h(v-s)
\]
After a change of variables (TODO: More detail), the two sides are equal. 

\section*{Problem 2}

We will prove that $(f*g)*h \neq f*(g*h)$ if * is correlation. First off, let $x=f*g$ and let $y=g*h$. \\
\[
x(t) = (f*g)(t) = \sum_{s=-\infty}^{\infty} f(t+s)g(s)
\]
\[
y(t) = (g*h)(t) = \sum_{s=-\infty}^{\infty} g(s)h(t+s)
\]
This means that
\[
(x*h)(t) = \sum_{v=-\infty}^{\infty} x(v)h(t+v) = \sum_{v=-\infty}^{\infty} \sum_{s=-\infty}^{\infty} f(v+s)g(s)h(t+v)
\]
Similarly
\[
(f*y)(t) = \sum_{v=-\infty}^{\infty} f(t+v)y(v) = \sum_{v=-\infty}^{\infty} \sum_{s=-\infty}^{\infty} f(t+v)g(s)h(v+s)
\]
After a change of variables (TODO: More detail), the two sides are definitely not equal. 

\section*{Problem 3}

Let $g_1(x) = g_2(y) = N(0,\sigma^2)$. Then
\[
g_1(x) = \frac{1}{\sqrt{2\pi \sigma^2}} e^{-\frac{x^2}{2\sigma^2}}
\]
\[
g_2(y) = \frac{1}{\sqrt{2\pi \sigma^2}} e^{-\frac{y^2}{2\sigma^2}}
\]
This means that
\[
g_1(x)g_2(y) = \frac{1}{2\pi \sigma^2} e^{-\frac{x^2 + y^2}{2\sigma^2}}
\]
This allows us to say that $g(x,y) = g_1(x)g_2(y)$

\section*{Problem 4}

For the spatial domain running time, we can think of convolution as for each pixel in the image, we apply the filter to it. \\
Thus there are $H \cdot W$ iterations of the filter\\
Each filter will take $M \cdot N$ time. \\
Thus the time complexity is $O(MNHW)$\\
\\
If we use FFT, this would be the procedure:\\
1. Convert the signal to frequency\\
2. Convert the filter to frequency \\
3. Do element wise multiplication of the two new vectors. \\
4. Convert the elements back. \\
\\
Step 1 will take $MN \cdot log(MN)$ time\\
Step 2 will take $HW \cdot log(HW)$ time\\
Step 3 will take $max(HW,MN)$ time. The identity of the max will not matter in the end since the previous step eclipses this one. \\
Step 4 will take $max(HW,MN) log( max(HW, MN) )$ time. Again which one is the max does not matter because of step 1 and 2. \\
\\
The total running time is thus $O(MN \cdot log(MN) + HW \cdot log(HW) )$ time. \\
Assuming that $HW$ is the max, the running time is $O(HW \cdot log(HW))$. \\
\\
If the filter is $f(x,y) = f(x)g(y)$ then we could do the following:\\
1. Compute the horizontal filter\\
2. Compute the vertical filter\\
3. Do point wise multiplication \\
\\
Step 1 will take $HWM$ time and step 2 will be $HWN$ time. \\
Step 3 will be $HW$ time. \\
Total running time is thus $O( HW(M+N) )$ time. \\
TODO: Improve the algorithm and its analysis for the case of a filter being that product. 

\section*{Problem 5}

Our two gaussians are as follows
\[
g_1(t) = \frac{1}{\sqrt{2\pi \sigma_1^2}} e^{-\frac{t^2}{2\sigma_1^2}}
\]
\[
g_2(t) = \frac{1}{\sqrt{2\pi \sigma_2^2}} e^{-\frac{t^2}{2\sigma_2^2}}
\]
The convolution formula we will use is the following
\[
g_3(t) = \int_{-\infty}^{\infty}{g_1(s)g_2(t-s) \, ds}
\]
\\
Page 6 of this paper has a good explanation
\begin{verbatim}
http://www.tina-vision.net/docs/memos/2003-003.pdf
\end{verbatim}
Let $F_1$ be the Fourier transform for $g_1$ and $F_2$ be the transform for $g_2$.\\
\[
F_1(t) = \int_{-\infty}^{\infty}{g_1(s) e^{-2\pi i s t} \, ds}
\]
\[
F_2(t) = \int_{-\infty}^{\infty}{g_2(s) e^{-2\pi i s t} \, ds}
\]
Expanding we have
\[
F_1(t) = \frac{1}{\sqrt{2 \pi \sigma_1^2}} \int_{-\infty}^{\infty}{e^{-\frac{s^2}{2\sigma_1^2}} e^{-2\pi i s t} \, ds}
\]
\[
F_1(t) = \frac{1}{\sqrt{2 \pi \sigma_2^2}} \int_{-\infty}^{\infty}{e^{-\frac{s^2}{2\sigma_2^2}} e^{-2\pi i s t} \, ds}
\]

%\begin{figure}[H]
%\centering
%\includegraphics[height=4in]{prob1plot.jpg}
%\caption{Probability of Class Labels with decision boundaries marked}
%\end{figure}


\end{document}








